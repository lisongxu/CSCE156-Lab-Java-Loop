\documentclass[12pt]{scrartcl}

\setlength{\parindent}{0pt}
\setlength{\parskip}{.25cm}

\usepackage{graphicx}

\usepackage{xcolor}

\definecolor{darkred}{rgb}{0.5,0,0}
\definecolor{darkgreen}{rgb}{0,0.5,0}
\usepackage{hyperref}
\hypersetup{
  letterpaper,
  colorlinks,
  linkcolor=red,
  citecolor=darkgreen,
  menucolor=darkred,
  urlcolor=blue,
  pdfpagemode=none,
  pdftitle={CSCE 156 Lab Handout},
  pdfauthor={Christopher M. Bourke},
  pdfsubject={},
  pdfkeywords={}
}

\definecolor{MyDarkBlue}{rgb}{0,0.08,0.45}
\definecolor{MyDarkRed}{rgb}{0.45,0.08,0}
\definecolor{MyDarkGreen}{rgb}{0.08,0.45,0.08}

\definecolor{mintedBackground}{rgb}{0.95,0.95,0.95}
\definecolor{mintedInlineBackground}{rgb}{.90,.90,1}

%\usepackage{newfloat}
\usepackage[newfloat=true]{minted}
\setminted{mathescape,
               linenos,
               autogobble,
               frame=none,
               framesep=2mm,
               framerule=0.4pt,
               %label=foo,
               xleftmargin=2em,
               xrightmargin=0em,
               startinline=true,  %PHP only, allow it to omit the PHP Tags *** with this option, variables using dollar sign in comments are treated as latex math
               numbersep=10pt, %gap between line numbers and start of line
               style=default, %syntax highlighting style, default is "default"
               			    %gallery: http://help.farbox.com/pygments.html
			    	    %list available: pygmentize -L styles
               bgcolor=mintedBackground} %prevents breaking across pages
               
\setmintedinline{bgcolor={mintedBackground}}
\setminted[text]{bgcolor={mintedBackground},linenos=false,autogobble,xleftmargin=1em}
%\setminted[php]{bgcolor=mintedBackgroundPHP} %startinline=True}
\SetupFloatingEnvironment{listing}{name=Code Sample}
\SetupFloatingEnvironment{listing}{listname=List of Code Samples}

\title{CSCE 156 -- Computer Science II}
\subtitle{Lab 2.0 - Conditionals \& Loops}
\author{~}
\date{~}

\begin{document}

\maketitle

\section*{Prior to Lab}

Review this laboratory handout prior to lab.

For Java:
\begin{enumerate}
  \item Read if-then-else tutorial: \\
  	\url{http://download.oracle.com/javase/tutorial/java/nutsandbolts/if.html}
  \item Read switch/case tutorial:\\
  	\url{http://download.oracle.com/javase/tutorial/java/nutsandbolts/switch.html}
  \item Read for loop tutorial:\\
  	\url{http://download.oracle.com/javase/tutorial/java/nutsandbolts/for.html}
  \item Read while/do while loop tutorial:\\
  	\url{http://download.oracle.com/javase/tutorial/java/nutsandbolts/while.html}
\end{enumerate}

For PHP:
\begin{enumerate}
  \item Refer to the Control Structures section in the PHP Manual: \\
    \url{http://www.php.net/manual/en/language.control-structures.php}
\end{enumerate}

\section*{Lab Objectives \& Topics}
Following the lab, you should be able to:
\begin{itemize}
  \item Use if-then-else statements to control the logical flow of the program.
  \item Use switch-case statement to control the logical flow of the program.
  \item Use for/while loops to implement repetition statements in your program.
  \item Write complex programs that require conditional logical statements and or loops.
\end{itemize}

\section*{Peer Programming Pair-Up}

To encourage collaboration and a team environment, labs will be
structured in a \emph{pair programming} setup.  At the start of
each lab, you will be randomly paired up with another student 
(conflicts such as absences will be dealt with by the lab instructor).
One of you will be designated the \emph{driver} and the other
the \emph{navigator}.  

The navigator will be responsible for reading the instructions and
telling the driver what to do next.  The driver will be in charge of the
keyboard and workstation.  Both driver and navigator are responsible
for suggesting fixes and solutions together.  Neither the navigator
nor the driver is ``in charge.''  Beyond your immediate pairing, you
are encouraged to help and interact and with other pairs in the lab.

Each week you should alternate: if you were a driver last week, 
be a navigator next, etc.  Resolve any issues (you were both drivers
last week) within your pair.  Ask the lab instructor to resolve issues
only when you cannot come to a consensus.  

Because of the peer programming setup of labs, it is absolutely 
essential that you complete any pre-lab activities and familiarize
yourself with the handouts prior to coming to lab.  Failure to do
so will negatively impact your ability to collaborate and work with 
others which may mean that you will not be able to complete the
lab.  

\section*{Getting Started}

Clone the project code for this lab from GitHub in Eclipse using the
URL, \url{https://github.com/cbourke/CSCE156-Lab02}.
Refer to Lab 1.0 for instructions on how to clone a project from GitHub.

For those with prior Java experience, do the PHP section.  For those
without prior Java experience, do the Java section.

\part*{PHP}

\section*{Conditionals \& Loops}

PHP provides standard control structures for conditionals and repetition.  Specifically, PHP provides the usual if-then-else statements and while, for, and do-while loops.  The syntax for these control structures should look familiar; some examples:

\begin{minted}{php}
if(condition1) {
  //DO SOMETHING
} else if(condition2) {
  //DO SOMETHING ELSE
} else {
  //OTHERWISE
}

for($i=0; $i<$n; $i++) {
  //DO SOMETHING
}

$i=0;
while($i<$n) {
  //DO SOMETHING
  $i++;
}

$i=0;
do{ 
  //DO SOMETHING
  $i++;
} while($i<$n);
\end{minted}

In addition, PHP provides a foreach-loop for iterating over elements in 
an array.  This is not just for convenience: in PHP arrays are associative 
so they are not necessarily indexed 0 thru n - 1; arrays may not even 
be indexed with integers!  Instead, array should be considered to be a
collection of key-value pairs.  The following examples illustrate the 
foreach loop's usage.

\begin{minted}{php}
foreach($array as $value) {
  print "$value \n";
}

foreach($array as $key => $value) {
  print "The key $key maps to the value $value\n";
}
\end{minted}

\section*{Activities}

\subsection*{Sum of Natural Numbers}

Natural numbers are the usual counting numbers; 1, 2, 3, \ldots.  In 
this exercise you will write several loops to compute the sum of 
natural numbers 1 thru $n$ where $n$ is read from the command line.  
You will also write a foreach loop to iterate over an array and 
process data.

\begin{enumerate}
  \item Open the \mintinline{text}{natural.php} source file.  The code 
	to read in $n$ has already been provided for you.  An array mapping 
	integer values 1 thru 10 to text values has also been created for you.
  \item Write a for-loop and a while-loop to compute the sum of natural 
  	numbers 1 thru $n$ and output the answer.
  \item Write a foreach loop to iterate over the elements (key/value pairs) 
	of the \mintinline{php}{$oneToTen} array.  As you iterate over the 
	elements you should sum the keys and concatenate the values to 
	formulate the following string (which you should output at the end 
	of the for-loop):
	
	\mintinline{text}{one + two + three + four + five + six + seven + eight + nine + ten = 55}
  \item Hand in your program using the webhandin and use the webgrader
    to verify your program works correctly.
\end{enumerate}
	
\subsection*{Child Tax Credit}

When filing for federal taxes, a credit is given to tax payers 
with dependent children according to the following rules.  The 
first dependent child younger than 18 is worth a \$1000.00 credit.  
Each dependent child younger than 18 after the first child is 
worth a \$500 tax credit each.  You will complete a PHP script 
to output a table of dependent children, how much each contributes 
to a tax credit, and a total child tax credit.  Your table should 
look something like the following.

\begin{minted}{text}
Child           Amount
Tommy (14)      $1000.00
Richard (12)    $500.00
Harold (21)     $0.00
Total Credit:   $1500.00
\end{minted}

\begin{enumerate}
  \item Open the \mintinline{text}{Child.php} and 
  	\mintinline{text}{ChildCredit.php} script files
  \item The \mintinline{php}{Child} class has already been defined 
  	and included in the \mintinline{text}{ChildCredit.php} script.  
	Note how the \mintinline{php}{Child} class is used; several 
	instances of children have been created and placed into an array.
  \item Write code to iterate over the array, compute the child tax 
	credits and output a table similar to the one above.  Note: to 
	call a method on an instance of the \mintinline{php}{Child} class, 
	use the following syntax: \mintinline{php}{$kid->getAge()}
  \item Answer the questions in your worksheet and demonstrate your 
  	working code to a lab instructor.
\end{enumerate}
	
\subsection*{Advanced Activity (Optional)}
	
Modify the Child Tax Credit program to output the data in a 
well-formatted HTML table.  Demonstrate your dynamic webpage 
to a lab instructor.


\newpage
\part*{Java}

\section*{Conditionals \& Loops}

Java provides standard control structures for conditionals and 
repetition.  Specifically, Java provides the usual if-then-else 
statements and while, for, and do-while loops.  The syntax for 
these control structures should look familiar; some examples:

\begin{minted}{java}
if(condition1) {
  //DO SOMETHING
} else if(condition2) {
  //DO SOMETHING ELSE
} else {
  //OTHERWISE
}

for(int i=0; i<n; i++) {
  //DO SOMETHING
}

int i=0;
while(i<n) {
  //DO SOMETHING
  i++;
}

int i=0;
do{ 
  //DO SOMETHING
  i++;
} while(i<n);
\end{minted}

In addition, Java provides a foreach-loop, also referred to as an 
\emph{enhanced for-loop}, for iterating over collections (classes 
that implement the \mintinline{java}{Iterable} interface) or elements 
in an array.  This feature is mostly for convenience.  The following 
examples illustrate this loop's usage.

\begin{minted}{java}
String arr[] = new String[10];
...
for(String s : arr) {
  System.out.println(s); 
}
\end{minted}

\section*{Activities}

\subsection*{Sum of Natural Numbers}

Natural numbers are the usual counting numbers; 1, 2, 3, \ldots.  In 
this exercise you will write several loops to compute the sum of 
natural numbers 1 thru $n$ where $n$ is read from the command line.  
You will also write an enhanced for-loop to iterate over an array and 
process data.

\begin{enumerate}
  \item Open the \mintinline{text}{Natural.java} source file.  The code 
	to read in $n$ has already been provided for you.  An array mapping 
	integer values 1 thru 10 to text values has also been created for you.
  \item Write a for-loop and a while-loop to compute the sum of natural 
  	numbers 1 thru $n$ and output the answer.
  \item Write a foreach loop to iterate over the elements (key/value pairs) 
	of the \mintinline{java}{zeroToTen} array.  As you iterate over the 
	elements,concatenate each string, delimited by a single space to a 
	result string and print the result at the end of the loop.  Your 
	result should look something like the following:
	
	\mintinline{text}{zero + one + two + three + four + five + six + seven + eight + nine + ten = 55}
  \item Hand in your program using the webhandin and use the webgrader
    to verify your program works correctly.
\end{enumerate}

\subsection*{Child Tax Credit}

When filing for federal taxes, a credit is given to tax payers 
with dependent children according to the following rules.  The 
first dependent child younger than 18 is worth a \$1000.00 credit.  
Each dependent child younger than 18 after the first child is 
worth a \$500 tax credit each.  You will complete a Java program
to output a table of dependent children, how much each contributes 
to a tax credit, and a total child tax credit.  Your table should 
look something like the following.

\begin{minted}{text}
Child           Amount
Tommy (14)      $1000.00
Richard (12)    $500.00
Harold (21)     $0.00
Total Credit:   $1500.00
\end{minted}

\begin{enumerate}
  \item Open the \mintinline{text}{Child.java} and 
  	\mintinline{text}{ChildCredit.java} source files
  \item The \mintinline{java}{Child} class has already been implemented
  	for you. Note how the \mintinline{java}{Child} class is used; several 
	instances of children have been created and placed into an array.
  \item Write code to iterate over the array, compute the child tax 
	credits and output a table similar to the one above.  Note: to 
	call a method on an instance of the \mintinline{java}{Child} class, 
	use the following syntax: \mintinline{java}{kid.getAge()}
  \item Answer the questions in your worksheet and demonstrate your 
  	working code to a lab instructor.
\end{enumerate}

\subsection*{Advanced Activity (Optional)}

Use the \mintinline{java}{String.format()} method to reformat the 
output of the Child Tax Credit program to print every piece of data 
in its own column.

\end{document}
